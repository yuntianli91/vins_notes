\section{KLT稀疏光流跟踪算法}
\label{append.a}
\section{基于四元数微分的IMU预积分模型}
\label{append.b}
\subsection{IMU噪声与零偏模型}
IMU的量测模型可表示为:
\begin{equation}
\left\{
\begin{aligned}
&\hat{\bm{a}}^b=\bm{a}^b+\bm{b}_a+\bm{R}_w^b\bm{g}^w+\bm{n}_a\\
&\hat{\bm{\omega}}^b=\bm{\omega}^b+\bm{b}_\omega+\bm{n}_\omega
\end{aligned}
\right.
\end{equation}
式中,$\hat{\bm{a}}^b$和$\hat{\bm{\omega}}^b$分别表示IMU的加速度和角速度测量信息;$\bm{a}^b$和$\bm{\omega}^b$分别为载体真实的加速度和角速度;$\bm{b}_a$和$\bm{b}_\omega$分别为加速度计和陀螺仪零偏,建模为随机游走过程:
\begin{equation}
\centering
\left\{
\begin{aligned}
&\bm{n}_{\bm{b}_a}\sim\mathcal{N}\left(\bm{0},\bm{\sigma}_{b_a}^2\right),\bm{n}_{\bm{b}_\omega}\sim\mathcal{N}\left(\bm{0},\bm{\sigma}_{b_\omega}^2\right)\\
&\dot{\bm{b}_a}=\bm{n}_{\bm{b}_a},\dot{\bm{b}_\omega}=\bm{n}_{\bm{b}_\omega}
\end{aligned}
\right.
\end{equation}
$\bm{R}_w^b$为世界坐标系到本体坐标系的坐标旋转矩阵;$\bm{g}^w$为世界坐标系下的重力加速度矢量;$\bm{n}_a\sim\mathcal{N}\left(\bm{0},\bm{\sigma}_a^2\right)$和$\bm{n}_\omega\sim\left(\bm{0},\bm{\sigma}_\omega^2\right)$为器件的高斯白噪声。
\subsection{四元数左乘与右乘}
假设一个四元数$\bm{q}_a=\left[\begin{array}{cccc}w_a & x_a & y_a & z_a\end{array}\right]^T=\left[\begin{array}{cc}w_a & \bm{v}_a\end{array}\right]^T$,左乘和右乘的矩阵形式分别为:
\begin{equation}
\begin{aligned}
L\left(\bm{q}_a\right)&=\left[\begin{matrix}
w_a & -x_a & -y_a & -z_a\\
x_a &  w_a & -z_a &  y_a\\
y_a &  z_a &  w_a & -x_a\\
z_a & -y_a &  x_a &  w_a
\end{matrix}\right]=w_a\bm{I}+\left[\begin{matrix}
0 & -\bm{v}_a^T\\
\bm{v}_a & \bm{v}_a^\wedge
\end{matrix}\right]\\
R\left(\bm{q}_a\right)&=\left[\begin{matrix}
w_a & -x_a & -y_a & -z_a\\
x_a &  w_a &  z_a & -y_a\\
y_a & -z_a &  w_a &  x_a\\
z_a &  y_a & -x_a &  w_a
\end{matrix}\right]=w_a\bm{I}+\left[\begin{matrix}
0 & -\bm{v}_a^T\\
\bm{v}_a & -\bm{v}_a^\wedge
\end{matrix}\right]
\end{aligned}
\end{equation}
若四元数的实部位于最后,即$\bm{q}_a=\left[\begin{array}{cccc}x_a & y_a & z_a & w_a\end{array}\right]^T=\left[\begin{array}{cc}\bm{v}_a & w_a\end{array}\right]^T$,则对应的左乘和右乘矩阵为:
\begin{equation}
\begin{aligned}
L\left(\bm{q}_a\right)&=\left[\begin{matrix}
  w_a & -z_a &  y_a & x_a\\
  z_a &  w_a & -x_a & y_a\\
 -y_a &  x_a &  w_a & z_a\\
 -x_a & -y_a & -z_a & w_a
\end{matrix}\right]=w_a\bm{I}+\left[\begin{matrix}
\bm{v}_a^\wedge & \bm{v}_a\\
-\bm{v}_a^T & 0
\end{matrix}\right]\\
R\left(\bm{q}_a\right)&=\left[\begin{matrix}
  w_a &  z_a & -y_a & x_a\\
 -z_a &  w_a &  x_a & y_a\\
  y_a & -x_a &  w_a & z_a\\
 -x_a & -y_a & -z_a & w_a
\end{matrix}\right]=w_a\bm{I}+\left[\begin{matrix}
-\bm{v}_a^\wedge & \bm{v}_a\\
-\bm{v}_a^T & 0
\end{matrix}\right]
\end{aligned}
\end{equation}
\subsection{四元数微分}
假设世界坐标系旋转到本体坐标系的旋转轴为$\bm{n}$,旋转角为$\theta$,则对应的旋转四元数为$\bm{q}=\cos\left(\frac{\theta}{2}\right)+\sin\left(\frac{\theta}{2}\right)\bm{n}$,该单位四元数对时间的导数为:
\begin{equation}
\centering
\frac{d\bm{q}}{dt}=-\frac{1}{2}\sin\left(\frac{\theta}{2}\right)\cdot\frac{d\theta}{dt}+\frac{d\bm{n}}{dt}\cdot\sin\left(\frac{\theta}{2}\right)+\bm{n}\cdot\frac{1}{2}\cos\left(\frac{\theta}{2}\right)\cdot\frac{d\theta}{dt}
\end{equation}
式中$-1=\bm{n}\cdot\bm{n}$,$\frac{d\bm{n}}{dt}=0$,$\frac{d\theta}{dt}=\omega^w_wb$,$\omega^w_wb$为世界坐标系下世界坐标系到本体系的旋转角速度。对上式进一步化简有:
\begin{equation}
\centering
\begin{aligned}
\frac{d\bm{q}}{dt}&=\frac{1}{2}\bm{n}\omega_{wb}^w\left(\cos\left(\frac{\theta}{2}\right)+\bm{n}\sin\left(\frac{\theta}{2}\right)\right)\\
&=\frac{1}{2}\bm{\omega}^w_{wb}\bm{q}
\end{aligned}
\end{equation}
由于陀螺测量得到的角速度为本体系下的角速度$\bm{\omega}_{wb}^b$,和世界坐标系下角速度的关系\footnote{假设四元数$\bm{q}$表示世界坐标系到本体坐标系的旋转,即$f^b=\bm{q}f^w\bm{q}^\ast$。而此处矢量本身并未旋转,只因坐标系旋转引起了坐标的变化,相当于坐标系旋转的逆过程,因此有$\bm{w}^b_{wb}=\bm{q}^\ast\bm{w}^w_{wb}\bm{q}$。}可由四元数表示$\bm{q}\bm{\omega}_{wb}^b\bm{q}^\ast=\bm{\omega}_{wb}^w$,带入上式有:
\begin{equation}
\centering
\frac{d\bm{q}}{dt}=\frac{1}{2}\bm{q}\bm{\omega}_{wb}^b\bm{q}^\ast\bm{q}
\end{equation}
利用单位四元数的性质$\bm{q}\bm{q}^{-1}=\bm{q}\bm{q}^\ast=\bm{I}$得到四元数微分和陀螺测量角速度之间的关系为:
\begin{equation}
\centering
\frac{d\bm{q}}{dt}=\frac{1}{2}\bm{q}\bm{\omega}_{wb}^b
\label{eq.append.b.6}
\end{equation}
上式写成矩阵形式为:
\begin{equation}
\centering
\begin{aligned}
\frac{d\bm{q}}{dt}&=\frac{1}{2}\left[\begin{matrix}
0 & -\omega_x & -\omega_y & -\omega_z \\
\omega_x & 0 & \omega_z & -\omega_y \\
\omega_y & -\omega_z & 0 & \omega_x \\
\omega_z & \omega_y & -\omega_x & 0 
\end{matrix}\right]\bm{q}\\
&=\frac{1}{2}\left[\begin{array}{cc}
0 & -{\bm{\omega}_{wb}^b}^T\\
\bm{\omega}_{wb}^b & -{\bm{\omega}_{wb}^b}^\wedge
\end{array}\right]\\
&=\frac{1}{2}\bm{\Omega}\left(\bm{\omega}_{wb}^b\right)\bm{q}
\end{aligned}
\end{equation}
该矩阵形式默认四元数的排列为实部在前。

此外,这里说明下四元数和坐标旋转矩阵之间的关系,假设$w$系旋转到$b$系对应的旋转四元数为$\bm{q}_w^b$,坐标旋转矩阵为$\bm{R}_w^b$,则对于任意一个矢量$\bm{a}$有$\bm{R}_w^b\bm{a}={\bm{q}_w^b}^\ast\bm{a}\bm{q}_w^b$,展开得:
\begin{equation}
\centering
\begin{aligned}
\bm{R}_w^b\bm{a}&=\left[\begin{array}{c}
\left(q_w^2+q_x^2-q_y^2-q_z^2\right)x + 2\left(q_wq_z+q_xq_y\right)y+2\left(q_xq_z-q_2q_y\right)z\\
2\left(q_xq_y-q_wq_z\right)x+\left(q_w^2-q_x^2+q_y^2-q_z^2\right)y+2\left(q_wq_x+q_yq_z\right)z\\
2\left(q_wq_y-q_xq_z\right)x+2\left(q_yq_z-q_wq_x\right)y+\left(q_w^2-q_x^2-q_y^2+q_z^2\right)z
\end{array}\right]\\
&=\left[\begin{array}{ccc}
\left(q_w^2+q_x^2-q_y^2-q_z^2\right) & 2\left(q_wq_z+q_xq_y\right) & 2\left(q_xq_z-q_2q_y\right)\\
2\left(q_xq_y-q_wq_z\right) & \left(q_w^2-q_x^2+q_y^2-q_z^2\right) & 2\left(q_wq_x+q_yq_z\right)\\
2\left(q_wq_y-q_xq_z\right) & 2\left(q_yq_z-q_wq_x\right) & \left(q_w^2-q_x^2-q_y^2+q_z^2\right)
\end{array}\right]\left[\begin{array}{c}x\\y\\z\end{array}\right]
\end{aligned}
\end{equation}
\subsection{四元数微分的毕卡算法}
式\ref{eq.append.b.6}所表示的四元数微分方程可采用毕卡级数求解,这里省略详细的推导过程,直接给出四元数的毕卡积分解:
\begin{equation}
\bm{q}_{t_{k+1}}=e^{\frac{1}{2}\int_{t_k}^{t_{k+1}}\bm{\Omega}\left(\bm{\omega_{wb}^b}\right)dt}\bm{q}_{t_k}
\label{eq.append.b.10}
\end{equation}
记$\Delta\bm{\Theta}=\int_{t_k}^{t_{k+1}}\bm{\Omega}\left(\bm{\omega_{wb}^b}\right)dt$,则:
\begin{equation}
\centering
\begin{aligned}
\Delta\Theta&=\int_{k}^{k+1}\left[\begin{matrix}
0 & -\omega_x & -\omega_y & -\omega_z \\
\omega_x & 0 & \omega_z & -\omega_y \\
\omega_y & -\omega_z & 0 & \omega_x \\
\omega_z & \omega_y & -\omega_x & 0 
\end{matrix}\right]dt\\
&\approx\left[\begin{matrix}
0 & -\Delta\theta_x & -\Delta\theta_y & -\Delta\theta_z\\
\Delta\theta_x & 0 & \Delta\theta_z & -\Delta\theta_y\\
\Delta\theta_y & -\Delta\theta_z & 0 & \Delta\theta_x\\
\Delta\theta_z & \Delta\theta_y & -\Delta\theta_x & 0
\end{matrix}\right]
\end{aligned}
\end{equation}
$\Delta\bm{\theta}=\left[\begin{array}{ccc}\Delta\theta_x & \Delta\theta_y&\Delta\theta_z\end{array}\right]$称为采样时间$\left[t_k, t_{k+1}\right]$内的角增量。将上式带入式\ref{eq.append.b.10}中并对其进行泰勒展开有:
\begin{equation}
\centering
\begin{aligned}
\bm{q}_{t_{k+1}}&=e^{\frac{1}{2}\Delta\bm{\Theta}}\bm{q}_{t_k}\\
&=\left(\bm{I}+\frac{\frac{1}{2}\Delta\bm{\Theta}}{1!}+\frac{\left(\frac{1}{2}\Delta\bm{\Theta}\right)^2}{2!}+\frac{\left(\frac{1}{2}\Delta\bm{\Theta}\right)^3}{3!}+\frac{\left(\frac{1}{2}\Delta\bm{\Theta}\right)^4}{4!}+o\left(\frac{1}{2}\Delta\bm{\Theta}\right)\right)\bm{q}_{t_k}\\
\end{aligned}
\label{eq.append.b.12}
\end{equation}
上式中,$o\left(\frac{1}{2}\Delta\bm{\Theta}\right)$为高阶小量。$\Delta\bm{\Theta}$的幂函数存在一定的规律:
\begin{equation}
\centering
\begin{aligned}
\Delta\bm{\Theta}^2&=-\Delta\Theta^2\bm{I}\\
\Delta\bm{\Theta}^3&=-\Delta\Theta^2\Delta\bm{\Theta}\\
\Delta\bm{\Theta}^4&=-\Delta\Theta^4\bm{I}\\
\Delta\bm{\Theta}^5&=\Delta\Theta^4\Delta\bm{\Theta}\\
\cdots
\end{aligned}
\end{equation}
将上式带入式\ref{eq.append.b.12}并合并包含和不包含$\Delta\bm{\Theta}$的同类项有:
\begin{equation}
\centering
\begin{aligned}
\bm{q}_{t_{k+1}}&=\left\{\bm{I}\left(1-\frac{\left(\frac{\Delta\theta}{2}\right)^2}{2!}+\frac{\left(\frac{\Delta\theta}{2}\right)^4}{4!}-\frac{\left(\frac{\Delta\theta}{2}\right)^6}{6!}+\cdots\right)+\frac{\Delta\bm{\Theta}}{2}\left(\frac{\left(\frac{\Delta\theta}{2}\right)}{1!}-\frac{\left(\frac{\Delta\theta}{2}\right)^3}{3!}+\frac{\left(\frac{\Delta\theta}{2}\right)^5}{5!}-\cdots\right)\frac{2}{\Delta\theta}\right\}\bm{q}_{t_k}\\
&=\left(\bm{I}\cos\frac{\Delta\theta}{2}+\Delta\bm{\Theta}\frac{\sin\frac{\Delta\theta}{2}}{\Delta\theta}\right)\bm{q}_{t_k}
\end{aligned}
\end{equation}
上式即为完整形式的四元数角增量毕卡算法。但实际应用中,为了避免计算三角函数,通常直接取近似解:
\begin{enumerate}
	\item 一阶近似毕卡算法:
	\begin{equation}
		\bm{q}_{t_k+1}=\left(\bm{I}+\frac{\Delta\bm{\Theta}}{2}\right)\bm{q}_{t_k}
	\end{equation}
	\item 二阶近似毕卡算法:
	\begin{equation}
	\bm{q}_{t_k+1}=\left[\bm{I}\left(1-\frac{\Delta\theta^2}{8}\right)+\frac{\Delta\bm{\Theta}}{2}\right]\bm{q}_{t_k}
	\end{equation}	
	\item 三阶近似毕卡算法:
	\begin{equation}
	\bm{q}_{t_k+1}=\left[\bm{I}\left(1-\frac{\Delta\theta^2}{8}\right)+\left(\frac{1}{2}-\frac{\Delta\theta^2}{48}\right)\Delta\bm{\Theta}\right]\bm{q}_{t_k}
	\end{equation}
	\item 四阶近似毕卡算法:
	\begin{equation}
	\bm{q}_{t_k+1}=\left[\bm{I}\left(1-\frac{\Delta\theta^2}{8}+\frac{\Delta\theta^4}{384}\right)+\left(\frac{1}{2}-\frac{\Delta\theta^2}{48}\right)\Delta\bm{\Theta}\right]\bm{q}_{t_k}
	\end{equation}
\end{enumerate}
其中,将一阶毕卡算法写成四元数乘法形式为:
\begin{equation}
	\bm{q}_{t_{k+1}}=\bm{q}_{t_k}\otimes\left[\begin{array}{c}
	1\\
	\frac{\Delta\theta_x}{2}\\
	\frac{\Delta\theta_y}{2}\\
	\frac{\Delta\theta_z}{2}
	\end{array}\right]\\
\end{equation}
进一步离散化有:
\begin{equation}
	\bm{q}_{t_{k+1}}=\bm{q}_{t_k}\otimes\left[\begin{array}{c}
	1\\
	\frac{1}{2}\bm{\omega}_{wb}^b\delta t
	\end{array}\right]
\end{equation}
\subsection{基于四元数的IMU预积分}
\subsubsection{连续时间形式}
载体的位置、速度和姿态四元数在时间区间$\left[t_k,t_{k+1}\right]$内的传播方程为:
\begin{equation}
\centering
\left\{
\begin{aligned}
&\bm{p}_{k+1}^w=\bm{p}_k^w+\bm{v}_k^w\Delta t_k+\iint_{t_i\in\left[t_k,t_{k+1}\right]}\left(\bm{R}_{b_{t_i}}^w\left(\hat{\bm{a}}_{t_i}-\bm{b}_a-\bm{n}_a\right)-\bm{g}^w\right)dt^2\\
&\bm{v}_{k+1}^w=\bm{v}_k^w+\int_{t_i\in\left[t_k,t_{k+1}\right]}\left(\bm{R}_{b_{t_i}}^w\left(\hat{\bm{a}}_{t_i}-\bm{b}_a-\bm{n}_a\right)-\bm{g}^w\right)dt\\
&\bm{q}_w^{b_{k+1}}=\bm{q}_w^{b_{k}}\otimes\int_{t_i\in\left[t_k,t_{k+1}\right]}\frac{1}{2}\bm{\Omega}\left(\hat{\bm{\omega}}_{wb}^w-\bm{b}_\omega-\bm{n}_\omega\right)\bm{q}_k^wdt
\end{aligned}
\right.
\label{eq.append.b.11}
\end{equation}
式中${\bm{q}_w^{b_{k+1}}}$为世界坐标系到本体坐标系的旋转四元数,$\otimes$表示四元数乘法,$\bm{\Omega}\left(\bm{\omega}\right)$为上一小节中由$\bm{\omega}$组成的齐次矩阵,$\bm{b}_k$和$\bm{b}_{k+1}$分别为第$k$帧和第$k+1$帧对应的体坐标系,$\bm{b}_{t_i}$为$t_i$时刻的体坐标系。

可以看出,IMU的由第$k$帧向第$k+1$帧的状态传播需要第$k$帧的旋转、位置和速度信息,这使得后端优化时,每当$k$帧位姿优化完后均需要重新计算$k+1$的相关变量;而若将\ref{eq.append.b.11}中右侧的积分项表示在$k$帧对应的本体系下,则该项不包含任何需要后端优化的位姿变量而只依赖于IMU的量测信息,更加便于后端优化处理。因此,增量形式的IMU状态传播方程为:
\begin{equation}
\centering
\left\{
\begin{aligned}
	&\bm{R}_w^{b_k}\bm{p}_{k+1}^w=\bm{R}_w^{b_k}\left(\bm{p}_k^w+\bm{v}_k^w\Delta t_k-\frac{1}{2}\bm{g}^w\Delta t_k^2\right)+\hat{\bm{\alpha}}_{k+1}^{b_k}\\
	&\bm{R}_w^{b_k}\bm{v}_{k+1}^w=\bm{R}_w^{b_k}\left(\bm{v}_k^w-\bm{g}^w\Delta t_k\right)+\hat{\bm{\beta}}_{k+1}^{b_k}\\
	&\bm{q}_{b_{k}}^w\otimes\bm{q}_w^{b_{k+1}}=\hat{\bm{\gamma}}_{k+1}^{b_k}
\end{aligned}
\right.
\end{equation}
式中,$\bm{\alpha}_{k+1}^{b_k}$、$\bm{\beta}_{k+1}^{b_k}$和$\bm{\gamma_{k+1}^{b_k}}$分别为$b_k$坐标系下第$k$和$k+1$帧之间IMU的预积分项:
\begin{equation}
\centering
\left\{
\begin{aligned}
&\hat{\bm{\alpha}}_{k+1}^{b_k}=\iint_{t_i\in\left[t_k,t_{k+1}\right]}\left(\bm{R}_{b_{t_i}}^w\left(\hat{\bm{a}}_{t_i}-\bm{b}_a-\bm{n}_a\right)\right)dt^2\\
&\hat{\bm{\beta}}_{k+1}^{b_k}=\int_{t_i\in\left[t_k,t_{k+1}\right]}\left(\bm{R}_{b_{t_i}}^w\left(\hat{\bm{a}}_{t_i}-\bm{b}_a-\bm{n}_a\right)\right)dt\\
&\hat{\bm{\gamma}}_{k+1}^{b_k}=\int_{t_i\in\left[t_k,t_{k+1}\right]}\frac{1}{2}\bm{\Omega}\left(\hat{\bm{\omega}}_{wb}^w-\bm{b}_\omega-\bm{n}_\omega\right)\bm{\gamma}_k^{t_i}dt
\end{aligned}
\right.
\label{eq.append.b.21}
\end{equation}
可以看出,\ref{eq.append.b.21}中的IMU预积分项只和实际IMU期间量测以及IMU零偏有关。当后端优化的IMU零偏变化时,若变化较小,则使用IMU预积分项关于零偏的一阶近似更新预积分项;否则,则按照式\ref{eq.append.b.21}重新积分IMU得到预积分项。IMU预积分关于零偏的线性化更新表达式可以表示为:
\begin{equation}
\left\{
\begin{aligned}
&\hat{\bm{\alpha}}_k^{k+1}\approx\hat{\bm{\alpha}}_k^{k+1}+\bm{J}^{\bm{\alpha}}_{\bm{b}_a}\delta\bm{b}_a+\bm{J}^{\bm{\alpha}}_{\bm{b}_\omega}\delta\bm{b}_\omega\\
&\hat{\bm{\beta}}_k^{k+1}\approx\hat{\bm{\beta}}_k^{k+1}+\bm{J}^{\bm{\beta}}_{\bm{b}_a}\delta\bm{b}_a+\bm{J}^{\bm{\beta}}_{\bm{b}_\omega}\delta\bm{b}_\omega\\
&\hat{\bm{\gamma}}_k^{k+1}=\hat{\bm{\gamma}}_k^{k+1}\otimes\bm{J}^{\bm{\gamma}}_{\bm{b}_\omega}\delta\bm{b}_\omega
\end{aligned}
\right.
\label{eq.append.b.13}
\end{equation}
\subsubsection{离散时间形式}
为了便于实际程序的执行,需要将上述连续形式的预积分方程改写为离散形式,由于噪声均值为零,预积分中可以忽略不计,因此两个IMU量测时刻$i$和$i+1$之间离散形式的IMU预积分方程为:
\begin{equation}
\centering
\left\{
\begin{aligned}
&\bm{p}_{i+1}^w=\bm{p}_i^w+\bm{v}_i^w\Delta t_i+\frac{1}{2}\overline{\bm{a}}_{t_i}\delta t^2\\
&\bm{v}_{i+1}^w=\bm{v}_i^w+\overline{\bm{a}}_{t_i}\delta t\\
&\bm{q}_w^{b_{i+1}}=\bm{q}_w^{b_{i}}\otimes\left[\begin{array}{c}1\\\frac{1}{2}\overline{\bm{\omega}}_{t_i}\delta t
\end{array}\right]
\end{aligned}
\right.
\end{equation}
式中,$\overline{\bm{a}}_{t_i}$和$\overline{\bm{\omega}}_{t_i}$分别表示离散区间内的离散加速度,具体表达形式由离散方法决定:
\begin{enumerate}
	\item 欧拉法:
	\begin{equation}
	\centering
	\left\{
	\begin{aligned}
	&\overline{\bm{a}}_{t_i}=\bm{R}_{b_{t_i}}^w\left(\hat{\bm{a}}_{t_i}-\bm{b}_a\right)-\bm{g}^w\\
	&\overline{\bm{\omega}}_{t_i}=\hat{\bm{\omega}}_{t_i}-\bm{b}_\omega
	\end{aligned}
	\right.
	\end{equation}
	\item 中值法:
	\begin{equation}
	\centering
	\left\{
	\begin{aligned}
	&\overline{\bm{a}}_{t_i}=\frac{1}{2}\left[\bm{R}_{b_{t_i}}^w\left(\hat{\bm{a}}_{t_i}-\bm{b}_a\right)+\bm{R}_{b_{t_{i+1}}}^w\left(\hat{\bm{a}}_{t_{i+1}}-\bm{b}_a\right)\right]-\bm{g}^w\\
	&\overline{\bm{\omega}}_{t_i}=\frac{1}{2}\left(\hat{\bm{\omega}}_{t_i}+\hat{\bm{\omega}}_{t_{i+1}}\right)-\bm{b}_\omega
	\end{aligned}
	\right.
	\end{equation}
\end{enumerate}
同理,可得到离散的增量形式IMU预积分方程为:
\begin{equation}
\centering
\left\{
\begin{aligned}
&\hat{\bm{\alpha}}_{i+1}^{b_k}=\hat{\bm{\alpha}}_i^{b_k}+\hat{\bm{\beta}}_i^{b_k}\delta t+\frac{1}{2}\overline{\bm{a}}_i\delta t^2\\
&\hat{\bm{\beta}}_{i+1}^{b_k}=\hat{\bm{\beta}}_i^{b_k}+\overline{\bm{a}}_i\delta t\\
&\hat{\bm{\gamma}}_{i+1}^{b_k}=\hat{\bm{\gamma}}_i^{b_k}\otimes\left[\begin{array}{c}1\\\frac{1}{2}\overline{\bm{\omega}}_i\delta t\end{array}\right]
\end{aligned}
\right.
\end{equation}
同样有欧拉法和中值法两种离散形式:
\begin{enumerate}
	\item 欧拉法:
	\begin{equation}
	\centering
	\left\{
	\begin{aligned}
	&\overline{\bm{a}}_i=\bm{R}\left(\hat{\bm{\gamma}}_i^{b_k}\right)\left(\hat{\bm{a}}_i-\bm{b}_a\right)\\
	&\overline{\bm{\omega}}_i=\hat{\bm{\omega}}_i-\bm{b}_\omega
	\end{aligned}
	\right.
	\end{equation}
	\item 中值法:
	\begin{equation}
	\centering
	\left\{
	\begin{aligned}
	&\overline{\bm{a}}_i=\frac{1}{2}\left[\bm{R}\left(\hat{\bm{\gamma}}_i^{b_k}\right)\left(\hat{\bm{a}}_i-\bm{b}_a\right)+\bm{R}\left(\hat{\bm{\gamma}}_{i+1}^{b_k}\right)\left(\hat{\bm{a}}_{i+1}-\bm{b}_a\right)\right]\\
	&\overline{\bm{\omega}}_i=\frac{1}{2}\left(\hat{\bm{\omega}}_i+\hat{\bm{\omega}}_{i+1}\right)-\bm{b}_\omega
	\end{aligned}
	\right.
	\end{equation}
\end{enumerate}
\subsection{IMU预积分误差动力学方程与协方差矩阵}
在VINS的优化中,由于采用马氏距离构建二乘函数,需要对状态的协方差矩阵进行更新,而协方差矩阵的一步预测依赖于状态的误差动力学方程,因此这里我们首先就IMU预积分项的误差动力学方程进行推导。涉及的状态量包括位置增量$\bm{\alpha}_{k+1}^{b_k}$、速度增量$\bm{\beta}_{k+1}^{b_k}$、姿态增量$\bm{\theta}_{k+1}^{b_k}$、加速度计零偏$\bm{b}_a$和陀螺仪零偏$\bm{b_\omega}$,其中由于陀螺器件测量为角增量,这里使用角增量$\bm{\theta}_{k+1}^{b_k}$代替四元数增量$\bm{\gamma}_{k+1}^{b_k}$。
\subsubsection{连续时间形式}
由于我们需要求解的是上述各增量的误差动力学方程,因此系统状态空间矢量可记为\footnote{由于增量表示的是$k+1$和$k$之间的相对值,这里为了书写方便,下标使用$k$而非前面$k+1$,同时改用下标$t$表示IMU测量时刻,以区别图像帧时刻$k$,但两者本质上是同一个量。}。
\begin{equation}
\delta\bm{x}=\left[\begin{array}{ccccccccc}
\delta\bm{\alpha}_t^{b_k} & \delta\bm{\beta}_t^{b_k} & \delta\bm{\theta}_t^{b_k} & \delta\bm{b}_a & \delta\bm{b}_\omega & \bm{n}_a & \bm{n}_\omega & \bm{n}_{\bm{b}_a} & \bm{n}_{\bm{b}_\omega}
\end{array}\right] 
\end{equation}
由于导数运算的顺序可以交换,即$\delta\dot{\bm{a}}=\dot{\delta\bm{a}}$,我们可以先求当前变量的时间导数,再求其对各状态量的偏导数。
\begin{enumerate}
	\item \textbf{位置增量误差}$\delta\bm{\alpha}_t^{b_k}$
	
	位置增量的时间导数为$\dot{\bm{\alpha}}_t^{b_k}=\bm{\beta}_t^{b_k}$,显然,除了偏导数$\delta\dot{\bm{\alpha}}_t^{b_k}/\delta\bm{\beta}_t^{b_k}=\bm{I}$外,位置增量的导数对其余状态增量的导数为零,即其余状态增量的变化对位置增量误差的导数没有贡献.
	\item \textbf{速度增量误差}$\delta\bm{\beta}_t^{b_k}$
	
	速度增量的时间导数为$\dot{\bm{\beta}}_t^{b_k}=\bm{R}_{b_t}^{b_k}\left(\hat{\bm{a}}_t-\bm{b}_a-\bm{n}_a\right)$,可以看出非零偏导数共有三项,分别为:
	\begin{equation}
	\begin{aligned}
	\frac{\delta\bm{\beta}_t^{b_k}}{\delta\bm{\theta}_k^{b_k}}&=-\bm{R}_{b_t}^{b_k}\left(\hat{\bm{a}}_t-\bm{b}_a-\bm{n}_a\right)\\
	&\approx-\bm{R}_{b_t}^{b_k}\left(\hat{\bm{a}}_t-\bm{b}_a\right)\\
	\frac{\delta\bm{\beta}_t^{b_k}}{\bm{b}_a}&=-\bm{R}_{b_t}^{b_k}\\
	\frac{\delta\bm{\beta}_t^{b_k}}{\bm{n}_a}&=-\bm{R}_{b_t}^{b_k}
	\end{aligned}
	\end{equation}
	其中,由于状态递推的本质是随机变量均值的传播,而$\bm{n}_a$是零均值的高斯白噪声,可作为零值处理。
	\item \textbf{角增量误差}$\delta\bm{\theta}_t^{b_k}$
	
	由于角增量的运算中涉及四元数的乘法,使用导数微分定义求解会比较复杂,这里我们换一种思路,假设无任何偏差的四元数时间导数和带各项扰动的四元数时间导数分别为$\dot{\bm{q}}_n$和$\dot{\bm{q}}_t$,则:
	\begin{equation}
	\begin{aligned}
	\dot{\bm{q}}_t&=\frac{1}{2}\underbrace{\bm{q}\otimes\delta\bm{q}}_{\delta\bm{\theta}}\otimes\left[\begin{array}{c}
	0\\\hat{\bm{\omega}}-\bm{b}_\omega-\bm{n}_\omega-\delta\bm{b}_\omega	
	\end{array}\right]\\
	\dot{\bm{q}}_n&=\frac{1}{2}\bm{q}\otimes\left[\begin{array}{c}
	0\\\hat{\bm{\omega}}-\bm{b}_\omega
	\end{array}\right]	
	\end{aligned}
	\end{equation}
	同时,$\dot{\bm{q}}_t$又可表示为:
	\begin{equation}
	\begin{aligned}
	\dot{\bm{q}}_t&=\frac{\partial\left(\bm{q}\otimes\delta\bm{q}\right)}{\partial t}=\dot{\bm{q}}\otimes\delta\bm{q}+\bm{q}\otimes\dot{\delta\bm{q}}\\
	&=\frac{1}{2}\bm{q}\otimes\left[\begin{array}{c}
	0\\\bm{\omega}-\bm{b}_\omega
	\end{array}\right]\otimes\delta\bm{q}+\bm{q}\otimes\dot{\delta\bm{q}}
	\end{aligned}
	\end{equation}
	令上述两式相等可得到:
	\begin{equation}
	\begin{aligned}
	&\frac{1}{2}\bm{q}\otimes\delta\bm{q}\otimes\left[\begin{array}{c}
	0\\\hat{\bm{\omega}}-\bm{b}_\omega-\bm{n}_\omega-\delta\bm{b}_\omega	
	\end{array}\right]=\frac{1}{2}\bm{q}\otimes\left[\begin{array}{c}
	0\\\bm{\omega}-\bm{b}_\omega
	\end{array}\right]\otimes\delta\bm{q}+\bm{q}\otimes\dot{\delta\bm{q}}\\
	\Rightarrow&\frac{1}{2}\delta\bm{q}\otimes\left[\begin{array}{c}
	0\\\hat{\bm{\omega}}-\bm{b}_\omega-\bm{n}_\omega-\delta\bm{b}_\omega	
	\end{array}\right]=\frac{1}{2}\left[\begin{array}{c}
	0\\\bm{\omega}-\bm{b}_\omega
	\end{array}\right]\otimes\delta\bm{q}+\dot{\delta\bm{q}}\\
	\Rightarrow&2\dot{\delta}\bm{q}=\delta\bm{q}\otimes\left[\begin{array}{c}
	0\\\hat{\bm{\omega}}-\bm{b}_\omega-\bm{n}_\omega-\delta\bm{b}_\omega	
	\end{array}\right]-\left[\begin{array}{c}
	0\\\bm{\omega}-\bm{b}_\omega
	\end{array}\right]\otimes\delta\bm{q}\\
	\Rightarrow&2\dot{\delta\bm{q}}=\mathcal{R}\left(\left[\begin{array}{c}
	0\\\hat{\bm{\omega}}-\bm{b}_\omega-\bm{n}_\omega-\delta\bm{b}_\omega	
	\end{array}\right]\right)\delta\bm{q}-\mathcal{L}\left(\left[\begin{array}{c}
	0\\\bm{n}_\omega+\delta\bm{b}_\omega
	\end{array}\right]\right)\delta\bm{q}\\
	\Rightarrow&2\dot{\delta\bm{q}}=\left[\begin{matrix}
		0 & \left(\bm{n}_\omega+\delta\bm{b}_\omega\right)^T\\
		-\left(\bm{n}_\omega-\delta\bm{b}_\omega\right) & -\left(2\hat{\bm{\omega}}-2\bm{b}_\omega-\bm{n}_\omega-\delta\bm{b}_\omega\right)^\wedge\\
		\end{matrix}\right]\left[\begin{array}{c}
		1\\\frac{\delta\bm{\theta}}{2}
		\end{array}\right]\\
	\end{aligned}	
	\end{equation}
	将$\dot{\delta\bm{q}}=\left[\begin{array}{c}
	0\\\frac{1}{2}\dot{\delta\bm{\theta}}
	\end{array}\right]$带入上式可以并忽略零均值项$\bm{n}_\omega$以及二阶小量$\delta\bm{b}_\omega\delta\bm{\theta}$可以得到:
	\begin{equation}
	\begin{aligned}
	\left[\begin{array}{c}
	0\\\dot{\delta\bm{\theta}}\end{array}\right]&=\left[\begin{matrix}
	0 & \left(\bm{n}_\omega+\delta\bm{b}_\omega\right)^T\\
	-\left(\bm{n}_\omega-\delta\bm{b}_\omega\right) & -\left(2\hat{\bm{\omega}}-2\bm{b}_\omega-\bm{n}_\omega-\delta\bm{b}_\omega\right)^\wedge\\
	\end{matrix}\right]\left[\begin{array}{c}
	1\\\frac{\delta\bm{\theta}}{2}
	\end{array}\right]\\
	\Rightarrow\dot{\delta\bm{\theta}}&\approx-\left(\hat{\bm{\omega}}-\bm{b}_\omega\right)^\wedge\delta\bm{\theta}-\bm{n}_\omega-\delta\bm{b}_\omega
	\end{aligned}
	\end{equation}
	\item \textbf{加速度计零偏误差}$\delta\bm{b}_a$:根据IMU的器件模型有$\delta\dot{\bm{b}}_a=\bm{n}_{\bm{b}_a}$。
	\item \textbf{陀螺零偏误差}$\delta\bm{b}_\omega$:同样根据IMU的器件模型有$\delta\dot{\bm{b}}_\omega=\bm{n}_{\bm{b}_\omega}$。
\end{enumerate}

至此,我们已经获得了所有非零均值状态量动力学方程,写成矩阵形式为:
\begin{equation}
\begin{aligned}
\left[\begin{array}{c}
\delta\dot{\bm{\alpha}}_t^{b_k}\\\delta\dot{\bm{\beta}}_t^{b_k}\\\delta\dot{\bm{\theta}}_t^{b_k}\\
\delta\dot{\bm{b}_a}\\\delta\dot{\bm{b}_\omega}
\end{array}\right]&=\left[\begin{matrix}
0 & \bm{I} & 0 & 0 & 0\\
0 & 0 & -\bm{R}_{b_t}^{b_k}\left(\hat{\bm{a}}_t-\bm{b}_a\right)^\wedge & -\bm{R}_{b_t}^{b_k} & 0\\
0 & 0 & -\left(\hat{\bm{\omega}}_t-\bm{b}_\omega\right)^\wedge & 0 & -\bm{I}\\
0 & 0 & 0 & 0 & 0\\
0 & 0 & 0 & 0 & 0
\end{matrix}\right]\left[\begin{array}{c}
\delta\bm{\alpha}_t^{b_k}\\\delta\bm{\beta}_t^{b_k}\\\delta\bm{\theta}_t^{b_k}\\
\delta\bm{b}_a\\\delta\bm{b}_\omega
\end{array}\right]\\
&+\left[\begin{matrix}
0 & 0 & 0 & 0\\
-\bm{R}_{b_t}^{b_k} & 0 & 0 & 0\\
0 & -\bm{I} & 0 & 0\\
0 & 0 & \bm{I} & 0\\
0 & 0 & 0 & \bm{I}
\end{matrix}\right]\left[\begin{array}{c}
\bm{n}_a\\\bm{n}_\omega\\\bm{n}_{\bm{b}_a}\\\bm{n}_{\bm{b}_\omega}
\end{array}\right]\\
\Rightarrow\dot{\delta\bm{x}}_t&=\bm{f}_t\delta\bm{x}_t+\bm{v}_t\bm{n}_t
\end{aligned}\end{equation}
根据导数的定义,可以将上式改写为递推形式:
\begin{equation}
\begin{aligned}
\frac{\delta\bm{x}_{t+1}-\delta\bm{x}_t}{\delta t}&=\bm{f}_t\delta\bm{x}_t+\bm{v}_t\bm{n}_t\\
\delta\bm{x}_{t+1}&=\delta\bm{x}_t+\bm{f}_t\delta\bm{x}_t\delta t + \bm{v}_t\bm{n}_t\delta t\\
&=\underbrace{\left(\bm{I}+\bm{f}_t\delta t\right)}_{\bm{F}_t}\delta\bm{x}_t+\underbrace{\bm{v}_t\delta t}_{\bm{V}_t}\bm{n}_t
\end{aligned}
\end{equation}
可以看出,上式是一个标准的状态空间模型,给出了随机变量均值的传播方程;同样,我们可以直接写出随机变量方差的传播方程:
\begin{equation}
	\begin{aligned}
	\bm{P}_{t+1}&=\bm{F}_t\bm{P}_t\bm{F}_t^T+\bm{V}_t\bm{Q}_t\bm{V}_t^T\\
	\bm{P}_0&=\bm{0}\\
	\bm{Q}_0&=\left[\begin{matrix}
	\sigma_a^2 & 0 & 0 & 0\\
	0 & \sigma_\omega^2 & 0 & 0\\
	0 & 0 & \sigma_{\bm{b}_a}^2 & 0\\
	0 & 0 & 0 & \sigma_{\bm{b}_\omega}^2
	\end{matrix}\right]
	\end{aligned}
\end{equation}
误差系数自身的雅克比矩阵为:
\begin{equation}\begin{aligned}
\bm{J}_{k+1}&=\prod\limits_{t\in\left[k,k+1\right]}\bm{F}_t\bm{J}_k\\
\bm{J}_k&=\bm{I}
\end{aligned}
\end{equation}
注意这里的下标为图像帧$k$而非IMU量测$t$。
\subsubsection{离散时间形式}
同样,这里我们基于连续时间形式的误差动力学给出离散形式的误差递推方程,由于程序中采用的是中值积分,这里仅推导中值积分下的离散误差方程
\begin{enumerate}
	\item \textbf{角增量误差}$\delta\bm{\alpha}_t^{b_k}$
	
	由于速度误差中耦合了角增量误差,这里我们先给出角增量误差的离散形式,中值点导数$\overline{\delta\dot{\bm{\theta}}}$为:
	\begin{equation}
	\begin{aligned}
	\delta\dot{\bm{\theta}}_t&=-\left(\hat{\bm{\omega}}_t-\bm{b}_\omega\right)^\wedge\delta\bm{\theta}_t-{\bm{n}_\omega}_t-\delta\bm{b}_\omega\\
	\delta\dot{\bm{\theta}}_{t+1}&=-\left(\hat{\bm{\omega}}_{t+1}-\bm{b}_\omega\right)^\wedge\delta\bm{\theta}_t-{\bm{n}_\omega}_{t+1}-\delta\bm{b}_\omega\\
	\overline{\delta\dot{\bm{\theta}}}&=\frac{\left(\delta\dot{\bm{\theta}}_t+\delta\dot{\bm{\theta}}_{t+1}\right)}{2}\\
	&=-\left(\frac{\hat{\bm{\omega}}_t+\hat{\bm{\omega}}_{t+1}}{2}-\bm{b}_\omega\right)^\wedge\delta\bm{\theta}_t-\frac{{\bm{n}_\omega}_t+{\bm{n}_\omega}_{t+1}}{2}-\delta\bm{b}_\omega
	\end{aligned}
	\end{equation}
	根据导数定义展开得到:
	\begin{equation}
	\begin{aligned}
	\frac{\delta\bm{\theta}_{k+1}-\delta\bm{\theta}_k}{\delta t}&=-\left(\frac{\hat{\bm{\omega}}_t+\hat{\bm{\omega}}_{t+1}}{2}-\bm{b}_\omega\right)^\wedge\delta\bm{\theta}-\frac{{\bm{n}_\omega}_t+{\bm{n}_\omega}_{t+1}}{2}-\delta\bm{b}_\omega\\
	\Rightarrow\delta\bm{\theta}_{t+1}&=\left[\bm{I}-\left(\frac{\hat{\bm{\omega}}_t+\hat{\bm{\omega}}_{t+1}}{2}-\bm{b}_\omega\right)^\wedge\delta t\right]\delta\bm{\theta}-\frac{{\bm{n}_\omega}_t+{\bm{n}_\omega}_{t+1}}{2}\delta t-\delta\bm{b}_\omega\delta t
	\end{aligned}
	\label{eq.delta_theta_discrete}
	\end{equation}
	
	\item \textbf{速度增量误差}$\delta\bm{\beta}_t^{b_k}$
	
	同样我们先求中值点的导数值$\overline{\delta\dot{\bm{\beta}}}$:
	\begin{equation}
	\begin{aligned}
	\delta\dot{\bm{\beta}}_t&=-\bm{R}_{b_t}^{b_k}\left(\hat{\bm{a}}_t-\bm{b}_a\right)^\wedge\delta\bm{\theta}_t-\bm{R}_{b_t}^{b_k}\delta\bm{b}_a-\bm{R}_{b_t}^{b_k}{\bm{n}_a}_t\\
	\delta\dot{\bm{\beta}}_{t+1}&=-\bm{R}_{b_{t+1}}^{b_k}\left(\hat{\bm{a}}_{t+1}-\bm{b}_a\right)^\wedge\delta\bm{\theta}_{t+1}-\bm{R}_{b_{t+1}}^{b_k}\delta\bm{b}_a-\bm{R}_{b_{t+1}}^{b_k}{\bm{n}_a}_{t+1}\\
	\overline{\delta\dot{\bm{\beta}}}&=\frac{\delta\dot{\bm{\beta}}_t+\delta\dot{\bm{\beta}}_{t+1}}{2}\\
	&=-\frac{1}{2}\bm{R}_{b_t}^{b_k}\left(\hat{\bm{a}}_t-\bm{b}_a\right)^\wedge\delta\bm{\theta}_t-\frac{1}{2}\bm{R}_{b_{t+1}}^{b_k}\left(\hat{\bm{a}}_{t+1}-\bm{b}_a\right)^\wedge\delta\bm{\theta}_{t+1}-\frac{1}{2}\left(\bm{R}_{b_t}^{b_k}+\bm{R}_{b_{t+1}}^{b_k}\right)\delta\bm{b}_a\\
	&-\frac{1}{2}\bm{R}_{b_t}^{b_k}{\bm{n}_a}_t-\frac{1}{2}\bm{R}_{b_{t+1}}^{b_k}{\bm{n}_a}_{t+1}
	\end{aligned}
	\end{equation}
	由于使用了中值积分,这里需要使用$t+1$时刻的角增量误差$\delta\bm{\theta}_{t+1}$,将式\ref{eq.delta_theta_discrete}带入上式并按照对应的状态量整理得:
	\begin{equation}
	\begin{aligned}
		\delta\dot{\bm{\beta}}_t&=-\frac{1}{2}\bm{R}_{b_t}^{b_k}\left(\hat{\bm{a}}_t-\bm{b}_a\right)^\wedge\delta\bm{\theta}_t\\
		&-\frac{1}{2}\bm{R}_{b_{t+1}}^{b_k}\left(\hat{\bm{a}}_{t+1}-\bm{b}_a\right)^\wedge\left\{\left[\bm{I}-\left(\frac{\hat{\bm{\omega}}_t+\hat{\bm{\omega}}_{t+1}}{2}-\bm{b}_\omega\right)^\wedge\delta t\right]\delta\bm{\theta}_t-\frac{{\bm{n}_\omega}_t+{\bm{n}_\omega}_{t+1}}{2}\delta t-\delta\bm{b}_\omega\delta t\right\}\\
		&-\frac{1}{2}\left(\bm{R}_{b_t}^{b_k}+\bm{R}_{b_{t+1}}^{b_k}\right)\delta\bm{b}_a-\frac{1}{2}\bm{R}_{b_t}^{b_k}{\bm{n}_a}_t-\frac{1}{2}\bm{R}_{b_{t+1}}^{b_k}{\bm{n}_a}_{t+1}\\
		%%%%%%%%%%%%%%%%%%%%%%%%%%%%%%%%%%%%%%%%%%%%%%%%%%%%%%%%%%%%%%%%%%%%%%%%
		&=\left\{-\frac{1}{2}\bm{R}_{b_t}^{b_k}\left(\hat{\bm{a}}_t-\bm{b}_a\right)^\wedge-\frac{1}{2}\bm{R}_{b_{t+1}}^{b_k}\left(\hat{\bm{a}}_{t+1}-\bm{b}_a\right)^\wedge\left[\bm{I}-\left(\frac{\hat{\bm{\omega}}_t+\hat{\bm{\omega}}_{t+1}}{2}-\bm{b}_\omega\right)^\wedge\delta t\right]\right\}\delta\bm{\theta}_t\\
		&\qquad\begin{aligned}
			&-\frac{1}{2}\left(\bm{R}_{b_t}^{b_k}+\bm{R}_{b_{t+1}}^{b_k}\right)\delta\bm{b}_a+\frac{\delta t}{2}\bm{R}_{b_{t+1}}^{b_k}\left(\hat{\bm{a}}_{t+1}-\bm{b}_a\right)^\wedge\delta\bm{b}_\omega\\
			&-\frac{1}{2}\bm{R}_{b_t}^{b_k}{\bm{n}_a}_t-\frac{1}{2}\bm{R}_{b_{t+1}}^{b_k}{\bm{n}_a}_{t+1}\\
			&+\frac{\delta t}{4}\bm{R}_{b_{t+1}}^{b_k}\left(\hat{\bm{a}}_{t+1}-\bm{b}_a\right)^\wedge{\bm{n}_\omega}_t+\frac{\delta t}{4}\bm{R}_{b_{t+1}}^{b_k}\left(\hat{\bm{a}}_{t+1}-\bm{b}_a\right)^\wedge{\bm{n}_\omega}_{t+1}\\
		\end{aligned}
	\end{aligned}
	\end{equation}
	
	同样根据导数定义可以得到速度增量误差的递推方程:
	\begin{equation}
	\begin{aligned}
		\delta\bm{\beta}_{t+1}=\delta\bm{\beta}_t&+\left\{-\frac{1}{2}\bm{R}_{b_t}^{b_k}\left(\hat{\bm{a}}_t-\bm{b}_a\right)^\wedge-\frac{1}{2}\bm{R}_{b_{t+1}}^{b_k}\left(\hat{\bm{a}}_{t+1}-\bm{b}_a\right)^\wedge\left[\bm{I}-\left(\frac{\hat{\bm{\omega}}_t+\hat{\bm{\omega}}_{t+1}}{2}-\bm{b}_\omega\right)^\wedge\delta t\right]\right\}\delta t\delta\bm{\theta}_t\\
		&-\frac{1}{2}\left(\bm{R}_{b_t}^{b_k}+\bm{R}_{b_{t+1}}^{b_k}\right)\delta t\delta\bm{b}_a+\frac{\delta t^2}{2}\bm{R}_{b_{t+1}}^{b_k}\left(\hat{\bm{a}}_{t+1}-\bm{b}_a\right)^\wedge\delta\bm{b}_\omega\\
		&-\frac{1}{2}\bm{R}_{b_t}^{b_k}\delta t{\bm{n}_a}_t-\frac{1}{2}\bm{R}_{b_{t+1}}^{b_k}\delta t{\bm{n}_a}_{t+1}\\
		&+\frac{\delta t^2}{4}\bm{R}_{b_{t+1}}^{b_k}\left(\hat{\bm{a}}_{t+1}-\bm{b}_a\right)^\wedge{\bm{n}_\omega}_t+\frac{\delta t^2}{4}\bm{R}_{b_{t+1}}^{b_k}\left(\hat{\bm{a}}_{t+1}-\bm{b}_a\right)^\wedge{\bm{n}_\omega}_{t+1}\\
	\end{aligned}
	\end{equation}
	
	\item \textbf{位置增量误差}$\delta\bm{\alpha}_t^{b_k}$
	中值点导数值为$\overline{\delta\bm{\alpha}}_t=\frac{\delta\bm{\beta}_t+\delta\bm{\beta}_{t+1}}{2}$,进一步可得位置增量误差递推方程为:
	\begin{equation}
	\delta\bm{\alpha}_{t+1}=\delta\bm{\alpha}_t+\frac{\delta\bm{\beta}_t+\delta\bm{\beta}_{t+1}}{2}\delta t
	\end{equation}
	带入速度增量误差表达式有:
	\begin{equation}
	\centering
	\begin{aligned}
	\delta\bm{\alpha}_{t+1}&=\delta\bm{\alpha}_t+\frac{\delta t}{2}\delta\bm{\beta}_t\\
	&+\frac{\delta t}{2}\delta\bm{\beta}_t+\left\{-\frac{1}{2}\bm{R}_{b_t}^{b_k}\left(\hat{\bm{a}}_t-\bm{b}_a\right)^\wedge-\frac{1}{2}\bm{R}_{b_{t+1}}^{b_k}\left(\hat{\bm{a}}_{t+1}-\bm{b}_a\right)^\wedge\left[\bm{I}-\left(\frac{\hat{\bm{\omega}}_t+\hat{\bm{\omega}}_{t+1}}{2}-\bm{b}_\omega\right)^\wedge\delta t\right]\right\}\frac{\delta t}{2}\delta t\delta\bm{\theta}_t\\
	&-\frac{1}{2}\left(\bm{R}_{b_t}^{b_k}+\bm{R}_{b_{t+1}}^{b_k}\right)\delta t\frac{\delta t}{2}\delta\bm{b}_a+\frac{\delta t^2}{2}\bm{R}_{b_{t+1}}^{b_k}\left(\hat{\bm{a}}_{t+1}-\bm{b}_a\right)^\wedge\frac{\delta t}{2}\delta\bm{b}_\omega\\
	&-\frac{1}{2}\bm{R}_{b_t}^{b_k}\frac{\delta t}{2}\delta t{\bm{n}_a}_t-\frac{1}{2}\bm{R}_{b_{t+1}}^{b_k}\frac{\delta t}{2}\delta t{\bm{n}_a}_{t+1}\\
	&+\frac{\delta t^2}{4}\bm{R}_{b_{t+1}}^{b_k}\left(\hat{\bm{a}}_{t+1}-\bm{b}_a\right)^\wedge\frac{\delta t}{2}{\bm{n}_\omega}_t+\frac{\delta t^2}{4}\bm{R}_{b_{t+1}}^{b_k}\left(\hat{\bm{a}}_{t+1}-\bm{b}_a\right)^\wedge\frac{\delta t}{2}{\bm{n}_\omega}_{t+1}\\
	\end{aligned}
	\end{equation}
	整理后得到:
	\begin{equation}
	\centering
	\begin{aligned}
	\delta\bm{\alpha}_{t+1}&=\delta\bm{\alpha}_t+\delta\bm{\beta}_t\delta t\\
	&+\left\{-\frac{1}{4}\bm{R}_{b_t}^{b_k}\left(\hat{\bm{a}}_t-\bm{b}_a\right)^\wedge\delta t^2-\frac{1}{4}\bm{R}_{b_{t+1}}^{b_k}\left(\hat{\bm{a}}_{t+1}-\bm{b}_a\right)^\wedge\left[\bm{I}-\left(\frac{\hat{\bm{\omega}}_t+\hat{\bm{\omega}}_{t+1}}{2}-\bm{b}_\omega\right)^\wedge\delta t\right]\delta t^2\right\}\delta\bm{\theta}_t\\
	&-\frac{1}{4}\left(\bm{R}_{b_t}^{b_k}+\bm{R}_{b_{t+1}}^{b_k}\right)\delta t^2\delta\bm{b}_a+\frac{1}{4}\bm{R}_{b_{t+1}}^{b_k}\left(\hat{\bm{a}}_{t+1}-\bm{b}_a\right)^\wedge\delta t^3\delta\bm{b}_\omega\\
	&-\frac{1}{4}\bm{R}_{b_t}^{b_k}\delta t^2{\bm{n}_a}_t-\frac{1}{4}\bm{R}_{b_{t+1}}^{b_k}\delta t^2{\bm{n}_a}_{t+1}\\
	&+\frac{1}{8}\bm{R}_{b_{t+1}}^{b_k}\left(\hat{\bm{a}}_{t+1}-\bm{b}_a\right)^\wedge\delta t^3{\bm{n}_\omega}_t+\frac{1}{8}\bm{R}_{b_{t+1}}^{b_k}\left(\hat{\bm{a}}_{t+1}-\bm{b}_a\right)^\wedge\delta t^3{\bm{n}_\omega}_{t+1}\\
	\end{aligned}
	\end{equation}	
	
	\item \textbf{加速度计零偏误差}$\delta\bm{b}_a$:根据IMU的器件模型有$\delta\dot{\bm{b}}_a=\bm{n}_{\bm{b}_a}$,递推形式为:
	${\delta\bm{b}_a}_{t+1}={\delta\bm{b}_a}_t+\bm{n}_{\bm{b}_a}\delta t$。
	\item \textbf{陀螺零偏误差}$\delta\bm{b}_\omega$:同样根据IMU的器件模型有$\delta\dot{\bm{b}}_\omega=\bm{n}_{\bm{b}_\omega}$。
	
	同样,这里我们将上面所有的增量误差递推方程整理为矩阵形式有:
	\begin{equation}
	\begin{aligned}
	\left[\begin{array}{c}
	\delta\bm{\alpha}_{t+1}\\\delta\bm{\theta}_{t+1}\\\delta\bm{\beta}_{t+1}\\
	\delta{\bm{b}_a}_{t+1}\\\delta{\bm{b}_\omega}_{t+1}
	\end{array}\right]&=
	\left[\begin{matrix}
	\bm{I} & f_{01} & \delta t\bm{I} & f_{03} & f_{04}\\
	 0 & f_{11} & 0 & 0 & -\delta t\bm{I}\\
	 0 & f_{21} & \bm{I} & f_{23} & f_{24} \\
	 0 & 0 & 0 & \bm{I} & 0 \\
	 0 & 0 & 0 & 0 & \bm{I}
	\end{matrix}\right]\left[\begin{array}{c}
	\delta\bm{\alpha}_t\\\delta\bm{\theta}_t\\\delta\bm{\beta}_t\\
	\delta{\bm{b}_a}_t\\\delta{\bm{b}_\omega}_t
	\end{array}\right]\\
	&+\left[\begin{matrix}
	v_{00} & v_{01} & v_{02} & v_{03} & 0 & 0\\
	0 & -\frac{\delta t}{2}\bm{I} & 0 & -\frac{\delta t}{2}\bm{I} & 0 & 0\\
	-\frac{\bm{R}_{b_t}^{b_k}\delta t}{2} & v_{21} & -\frac{\bm{R}_{b_{t+1}}^{b_k}\delta t}{2} & v_{23} & 0 & 0\\
	0 & 0 & 0 & 0 & \delta t\bm{I} & 0\\
	0 & 0 & 0 & 0 & 0 & \delta t\bm{I}
	\end{matrix}\right]\left[\begin{array}{c}
	{\bm{n}_a}_t\\{\bm{n}_\omega}_t\\{\bm{n}_a}_{t+1}\\{\bm{n}_\omega}_{t+1}\\
	\bm{n}_{\bm{b}_a}\\\bm{n}_{\bm{b}_\omega}
	\end{array}\right]
	\end{aligned}
	\end{equation}
	$\bm{F}$矩阵中对应元素具体表达式如下:
	\begin{equation}
	\begin{aligned}	
	f_{01}&=\left\{-\frac{1}{4}\bm{R}_{b_t}^{b_k}\left(\hat{\bm{a}}_t-\bm{b}_a\right)^\wedge\delta t^2-\frac{1}{4}\bm{R}_{b_{t+1}}^{b_k}\left(\hat{\bm{a}}_{t+1}-\bm{b}_a\right)^\wedge\left[\bm{I}-\left(\frac{\hat{\bm{\omega}}_t+\hat{\bm{\omega}}_{t+1}}{2}-\bm{b}_\omega\right)^\wedge\delta t\right]\delta t^2\right\}\\
	f_{03}&=-\frac{1}{4}\left(\bm{R}_{b_t}^{b_k}+\bm{R}_{b_{t+1}}^{b_k}\right)\delta t^2\\
	f_{04}&=\frac{1}{4}\bm{R}_{b_{t+1}}^{b_k}\left(\hat{\bm{a}}_{t+1}-\bm{b}_a\right)^\wedge\delta t^3\\
	f_{11}&=\left[\bm{I}-\left(\frac{\hat{\bm{\omega}}_t+\hat{\bm{\omega}}_{t+1}}{2}-\bm{b}_\omega\right)^\wedge\delta t\right]\\
	f_{21}&=\left\{-\frac{1}{2}\bm{R}_{b_t}^{b_k}\left(\hat{\bm{a}}_t-\bm{b}_a\right)^\wedge-\frac{1}{2}\bm{R}_{b_{t+1}}^{b_k}\left(\hat{\bm{a}}_{t+1}-\bm{b}_a\right)^\wedge\left[\bm{I}-\left(\frac{\hat{\bm{\omega}}_t+\hat{\bm{\omega}}_{t+1}}{2}-\bm{b}_\omega\right)^\wedge\delta t\right]\right\}\delta t\\
	f_{23}&=-\frac{1}{2}\left(\bm{R}_{b_t}^{b_k}+\bm{R}_{b_{t+1}}^{b_k}\right)\delta t\\
	f_{24}&=\frac{1}{2}\bm{R}_{b_{t+1}}^{b_k}\left(\hat{\bm{a}}_{t+1}-\bm{b}_a\right)^\wedge\delta t^2
	\end{aligned}
	\end{equation}
	$\bm{V}$矩阵中对应元素具体表达式如下:
	\begin{equation}
	\begin{aligned}
	v_{00}&=-\frac{1}{4}\bm{R}_{b_t}^{b_k}\delta t^2\\
	v_{01}&=\frac{1}{8}\bm{R}_{b_{t+1}}^{b_k}\left(\hat{\bm{a}}_{t+1}-\bm{b}_a\right)^\wedge\delta t^3\\
	v_{02}&=-\frac{1}{4}\bm{R}_{b_{t+1}}^{b_k}\delta t^2\\
	v_{03}&=\frac{1}{8}\bm{R}_{b_{t+1}}^{b_k}\left(\hat{\bm{a}}_{t+1}-\bm{b}_a\right)^\wedge\delta t^3\\
	v_{21}&=\frac{1}{4}\bm{R}_{b_{t+1}}^{b_k}\left(\hat{\bm{a}}_{t+1}-\bm{b}_a\right)^\wedge\delta t^2\\
	v_{23}&=\frac{1}{4}\bm{R}_{b_{t+1}}^{b_k}\left(\hat{\bm{a}}_{t+1}-\bm{b}_a\right)^\wedge\delta t^2
	\end{aligned}
	\end{equation}
\end{enumerate}